\input ../talk-header.tex
\title{Python}
\subtitle{Introduction}

% If you wish to uncover everything in a step-wise fashion, uncomment
% the following command: 
%\beamerdefaultoverlayspecification{<+->}

\begin{document}

\begin{frame}
  \titlepage
\end{frame}

\begin{frame}{}
  \vphrase{scikit-learn}
\end{frame}

\begin{frame}{Datasets (1)}
  \texttt{import sklearn as sk}\\
  \texttt{from sklearn import datasets}\\[2mm]
  \texttt{iris = datasets.load\_iris()}
  \texttt{digits = datasets.load\_digits()}

  \bigskip
  Look at
  \begin{itemize}
  \item  \blue{\texttt{iris.data.size}}
  \item  \blue{\texttt{iris.data.shape}}
  \end{itemize}
\end{frame}

\begin{frame}{Datasets (2)}
  \texttt{load\_diabetes}\\
  \texttt{load\_boston}\\
  \texttt{load\_linnerud}\\
  \texttt{load\_wine}\\
  \texttt{load\_breast\_cancer}
\end{frame}

\begin{frame}{Datasets (3)}
  \begin{itemize}
  \item \texttt{data} : \texttt{n\_samples}, \texttt{n\_features}
  \item \texttt{target} : response variables
  \item \texttt{target\_names}
  \item \texttt{DESCR} : try \texttt{print()}
  \end{itemize}
\end{frame}

\begin{frame}{Classification}
  \texttt{from sklearn import svm}\\[1mm]
  \texttt{clf = svm.SVC(gamma=0.001, C=100.)}\\
  \texttt{clf.fit(digits.data[:-1], digits.target[:-1])}\\
  \texttt{clf.predict(digits.data[-1:])}
\end{frame}

\begin{frame}{matplotlib}

  {\tt
    import numpy as np \\
    \%matplotlib inline     {\purple{\small{\# Optional: Jupyter notebook}}} \\
    from matplotlib import pyplot as plt \\[4mm]
    x = np.arange(1, 11) \\
    y = 2 * x + 5 \\
    plt.title("Matplotlib demo") \\
    plt.xlabel("x axis caption") \\
    plt.ylabel("y axis caption") \\
    plt.plot(x, y) \\
    plt.show()
  }
\end{frame}

\begin{frame}{matplotlib}

  {\tt
    import numpy as np \\
    \%matplotlib inline     {\purple{\small{\# Optional: Jupyter notebook}}} \\
    from matplotlib import pyplot as plt \\[4mm]
    x = np.arange(1, 11) \\
    y = 2 * x + 5 \\
    plt.title("Matplotlib demo") \\
    plt.xlabel("x axis caption") \\
    plt.ylabel("y axis caption") \\
    plt.plot(x, y, 'dr') \\
    plt.show()
  }
\end{frame}

\begin{frame}{}

  {\tt
    import numpy as np \\
    import matplotlib.pyplot as plt  \\[4mm]
    x = np.arange(0, 3 * np.pi, 0.1) \\
    y = np.sin(x) \\
    plt.title("sine wave form") \\
    plt.plot(x, y, 'g') \\
    plt.show() 
  }
\end{frame}

\begin{frame}{}

  {\tt\small
    import numpy as np \\
    import matplotlib.pyplot as plt  \\[4mm]

    x = np.arange(0, 3 * np.pi, 0.1) \\
    y\_sin = np.sin(x) \\
    y\_cos = np.cos(x) \\[2mm]

    \purple{\# Set up a subplot grid that has height 2 and width 1,}\\
    \purple{\# and set the first such subplot as active.}\\
    plt.subplot(2, 1, 1)\\
    \purple{\# Make the first plot}\\
    plt.plot(x, y\_sin) \\
    plt.title('Sine')  \\
    \purple{\# Set the second subplot as active, and make the second plot.}\\
    plt.subplot(2, 1, 2) \\
    plt.plot(x, y\_cos) \\
    plt.title('Cosine') \\[2mm]
    plt.show()
  }
\end{frame}

\begin{frame}{}

  {\tt
    x = [5,8,10] \\
    y = [12,16,6]  \\[2mm]
    x2 = [6,9,11] \\
    y2 = [6,15,7] \\
    plt.bar(x, y, align = 'center') \\
    plt.bar(x2, y2, color = 'g', align = 'center') \\
    plt.title('Bar graph') \\
    plt.ylabel('Y axis') \\
    plt.xlabel('X axis')  \\[2mm]
    plt.show()
  }
\end{frame}

\begin{frame}
  \vfill
  \bf{\large\blue{\url{https://matplotlib.org/tutorials/introductory/sample_plots.html}}}
  \vfill
\end{frame}

\begin{frame}{matplotlib}
  \vphrase{\url{https://matplotlib.org/gallery/}}
\end{frame}

\begin{frame}{seaborn}

  {\tt
    import seaborn as sns \\[4mm]
    tips = sns.load\_dataset("tips")
    sns.relplot(x="total\_bill", y="tip", data=tips)\\[2mm]
    plt.show()
  }
\end{frame}

\begin{frame}{seaborn}
  
  {\tt sns.relplot(x="total\_bill", y="tip", \\
    \hspace{1cm} hue="smoker", data=tips)}
\end{frame}

\begin{frame}{seaborn}
  
  {\tt sns.relplot(x="total\_bill", y="tip",\\
    \hspace{1cm} hue="smoker", style="smoker",
    data=tips);
  }
\end{frame}

\begin{frame}{seaborn}

  {\tt
    sns.relplot(x="total\_bill", y="tip",\\
    \hspace{1cm} size="size", sizes=(15, 200), data=tips);
  }
\end{frame}

\begin{frame}{seaborn and random walks}

  {\tt
    df = pd.DataFrame(dict(time=np.arange(500),\\
        \hspace{2cm} value=np.random.randn(500).cumsum()))\\
    g = sns.relplot(x="time", y="value", kind="line", data=df)\\
    g.fig.autofmt\_xdate()\\
  }
\end{frame}

\begin{frame}{More random walks}

  {\tt
    df = pd.DataFrame(np.random.randn(500, 2).\\
    \hspace{1cm} cumsum(axis=0), columns=["x", "y"])
    sns.relplot(x="x", y="y", sort=False, \\
    \hspace{1cm} kind="line", data=df);
  }
\end{frame}

\begin{frame}{More seaborn examples}

  \blue{\bf\url{https://seaborn.pydata.org/tutorial/relational.html}}
\end{frame}

\begin{frame}{Seaborn and categorical data}

  \blue{\bf\url{https://seaborn.pydata.org/tutorial/categorical.html}}
\end{frame}

\begin{frame}{Seaborn and distributions}
  \blue{\bf\url{https://seaborn.pydata.org/tutorial/distributions.html}}
\end{frame}

\begin{frame}{Pour la prochaine fois}
  \vphrase{la même chose mais plus sophistiqué}

  \centerline{\url{J2/travail-pour-la-prochaine-fois.txt}}

  \bigskip
  \centerline{N'oubliez pas le tutoriel d'Addfor.}
\end{frame}

%%%%%%%%%%%%%%%%%%%%%%%%%%%%%%%%%%%%%%%%%%%%%%%%%%%%%%%%%%%%%%%%%%%%%%
%\talksection{Break}

\begin{frame}
  \frametitle{Questions?}
\end{frame}

\end{document}
