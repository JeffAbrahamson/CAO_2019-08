\input ../talk-header.tex
\title{Python}
\subtitle{Introduction}

% If you wish to uncover everything in a step-wise fashion, uncomment
% the following command: 
%\beamerdefaultoverlayspecification{<+->}

\begin{document}

\begin{frame}
  \titlepage
\end{frame}

\begin{frame}
  \vphrase{R, python}
\end{frame}

\begin{frame}
  \vphrase{statistics, data science, machine learning}
\end{frame}

\begin{frame}
  \vphrase{Hello, world}
\end{frame}

\begin{frame}{Indentation}
  \begin{itemize}
  \item Try to be like pseudo-code
  \item Consistency is the key
  \item Custom is 4 spaces, but there is variation
  \end{itemize}
\end{frame}

\begin{frame}
  \begin{itemize}
  \item import
  \item help, ?
  \end{itemize}
\end{frame}

\begin{frame}{Python}
  Common types:
  \begin{itemize}
  \item int
  \item float
  \item str
  \item list, tuple
  \item dict, set
  \item NoneType
  \end{itemize}
\end{frame}

\begin{frame}{Control flow}
  \begin{itemize}
  \item if, elif, else
  \item for, for in
  \item while
  \item range, zip, items
  \item return
  \end{itemize}
\end{frame}

\begin{frame}{Logic}
  \begin{itemize}
  \item True
  \item False
  \item and, or, not
  \end{itemize}
\end{frame}

\begin{frame}{Structure access}
  \begin{itemize}
  \item list elements, slices, stride
  \item list comprehensions
  \item and for dicts
  \item strings look like lists
  \item in
  \item sort() vs .sort()
  \end{itemize}
\end{frame}

\begin{frame}{functions}
  \begin{itemize}
  \item def
  \item parameters (and default, and named)
  \item comments
  \end{itemize}
\end{frame}

\begin{frame}{exercise}
  Write the function `find` to find a character in a string.

  For example, \texttt{find("le chien est bleu", "t")}.
\end{frame}

\begin{frame}{strings}
  \textit{try asking for help on a string}

  Some useful operations:
  \begin{itemize}
  \item find
  \item startswith
  \item endswith
  \item in
  \item +
  \end{itemize}
\end{frame}

\begin{frame}{exercise}
  Write a function that removes the nth element of a list.
\end{frame}

\begin{frame}{exercise}
  Write a function that generates a dict of the first 20 integers
  mapped to their squares.
\end{frame}

\begin{frame}{modules}
  \begin{itemize}
  \item random
  \item time
  \item datetime
  \item math
  \end{itemize}
\end{frame}

\begin{frame}{exercise}
  Write \texttt{Fib(n)} to generate the first n Fibonacci numbers.
\end{frame}

\begin{frame}{exercise}
  Write \texttt{Fib(n)} to generate the first n Fibonacci numbers.

  Now do it better.\\
  \texttt{\%timeit}
\end{frame}

\begin{frame}{Review}
  \url{https://github.com/addfor/tutorials}\\[2mm]

  \vspace{1mm}
  \hspace{5mm} II. Python Basic Concepts
\end{frame}

\begin{frame}{pandas}
  \vphrase{Introduction au pandas}
\end{frame}

\begin{frame}{Pour la prochaine fois}
  \phrase{pandas}

  \vspace{1cm}
  \centerline{titanic}

  \bigskip
  \centerline{N'oubliez pas le tutoriel d'Addfor.}
\end{frame}

%%%%%%%%%%%%%%%%%%%%%%%%%%%%%%%%%%%%%%%%%%%%%%%%%%%%%%%%%%%%%%%%%%%%%%
%\talksection{Break}

\begin{frame}
  \frametitle{Questions?}
\end{frame}

\end{document}
